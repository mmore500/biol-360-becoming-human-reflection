Watching ``Becoming Human,'' I was particularly fascinated by the emphasis placed by researchers on the extension of childhood.
The adolescent and child fossils found --- Turkana boy, a Neanderthal child, and Selam  --- were heralded as hugely important to understanding human evolution.
These fossils revealed that non-human hominids tended to have physiologically shorter childhoods.
For example, Turkana boy was beginning to reach sexual maturity at around age eight.
\textit{Homo sapiens} don't reach this stage until approximately age twelve.
However, the age of sexual maturity does not mesh well with our modern conception of childhood.
We consider individuals far past that age to be ``dependent'' or ``kids.''
In the United States, childhood legally extends until the age of eighteen.
In practice, we consider childhood to extend much further past that milestone.
For instance, it is not uncommon to hear persons aged up to 22 or 23 referred to as ``college kids.''
In considering my own career trajectory, I believe that an argument can be made that aspects of childhood might extend even further in some modern cases.

Next year, I will be entering a PhD program in the sciences.
This setting is centered around the social group of ``the laboratory'' or the ``research group.'' The research group is centered around the Primary Investigator, who serves a patriarchial or matriarchial role in the group in many respects, dispensing wisdom, mentorship, and wielding power over the resources that graduate students depend on to survive.
The patriarchial or matriarchial role of the PI is often considered in a very literal sense, with academics constructing ``academic pedigrees'' or ``academic family trees'' to trace their own lineage.
Considering graduate education as a further extension of some aspects of childhood, we now see that for some modern \textit{Homo sapiens} the social dynamics of childhood extend on to age 28 or beyond.

So, it seems that, at least in some cases, humans are continuing to extend aspects of childhood by leaps and bounds.
Importantly, this is taking place through cultural forces and not, it seems, through physiological changes exclusively.
What I'm really curious about, though, in relation to this apparent social extension of childhood is teasing apart the physiological and social aspects of childhood.
By the social aspects of childhood, I mean a relationship between two individuals in which the child-like individual seeks approval, guidance, and resources for sustenance from the person in the adult-like role and the person in the adult-like role is personally invested in the future success of the individual in the child-like role.
Importantly, this the social relationship at the center of childhood is not expected by both parties to last for an indefinite amount of time.
Instead, the child-like individual is expected to take on increasing responsibility for his or her own wellbeing and eventually become at least mostly independent from the control and support of the adult-like individual.
Does the social condition of childhood serve to support the the cultural transmission and intellectual/cranial development that evolutionary biologists believe its physiological extension in human evolution has allowed?
That is, are relationships of unequal power between individuals serving a child-like roles and an individual serving an adult-like roles necessary to support the transmission of cultural information between the two individuals and the intellectual development of the person in the child-like role?
I am especially curious if the social condition of childhood induces physiological changes in the individuals in the child-like and adult-like roles that mimic the physiological conditions of childhood and parenthood, respectively.
Do individuals engaged in a parent-child-like social relationship exhibit hormonal states, brain plasticity properties, etc. similar to those experienced by individuals engaged in a literal parent-child relationship?

Unrelated to my curiosity about the extension of childhood, the series ``Becoming Human'' also raised a few questions about popular-audience scientific media.
I am curious how producers choose to strike a balance between acknowledging scientific controversy and presenting a cohesive, compelling narrative to their viewers.
Although some points of contention were acknowledged in passing, I am highly suspicious that many of the theories in the film are likely not as widely or firmly accepted as perhaps a naive audience would be led to believe.
For example, one researcher claimed that sitting around the fire in large part led to the development of social aspects of humanity.
This seems a rather dubious claim, one for which no direct evidence was provided.
I am also curious about stylistic choices made by the producers of ``Becoming Human.''
The constantly moving camera (i.e. zooming through double speed rotations around individuals dusting off a rock with a paintbrush in Africa, etc.) and the heavy use (and, especially, reuse) of CGI visualization clips struck me as a bit distracting and gimmicky.
I wonder how effective these techniques are at capturing a general audience's attention and, especially, if it is possible for filmmakers to cater effectively to both the scientific and general audiences. Is it possible to produce a film that scientific viewers would take completely seriously that a general audience wouldn't sleep through or be overly confused by? Or are the tastes and of these audiences inherently mutually exclusive? I think that, for example, National Geographic pretty consistently does a very effective job of this in print, but I have yet to see a film accomplish this balancing act.